\section{Аннотация}


\subsection{Русский}

\subsubsection{Основная цель}
Исследование алгоритмов, позволяющих определить кратчайший путь между двумя вершинами в графе. Их практическое применение на примере карты с препятствиями. 

\subsubsection{Задачи}
Реализация программы на любом языке программирования, визуализирующей работу рассматриваемых алгоритмов поиска кратчайшего пути.

\subsubsection{Результаты}
Реализована программа на языке JavaScript, позволяющая:
\begin{itemize}
  \item Найти кратчайший путь между двумя точками на карте, если он существует.
  \item Оценить время работы алгоритма, количество обработанных вершин, существование оптимального пути и его длину.
  \item Изучить и сравнить алгоритм A* и алгоритм Дейкстры на основе визуализации их работы.
  \item Изменять размер карты и её топологию.
  \item Выбирать начальную и конечную точку, если они не являются препятствиями.
\end{itemize}

\subsubsection{Рекомендации}
\begin{itemize}
  \item Имеется начальная карта с размещёнными препятствиями, начальной точкой и конечной точкой пути.
  \item Начальная и конечная точки должны находятся внутри карты.
  \item Конечная точка должна быть пустой.
\end{itemize}



\subsection{English (WIP)}

\subsubsection{The main objective}
Исследование алгоритмов, позволяющих определить оптимальный путь между двумя вершинами в графе.

\subsubsection{Tasks}
Реализация программы на любом языке программирования, визуализирующей работу рассматриваемых алгоритмов поиска оптимального пути.

\subsubsection{Results}
Реализована программа на языке JavaScript, позволяющая:
\begin{itemize}
  \item Найти кратчайший путь между двумя точками на карте, если он существует.
  \item Оценить время работы алгоритма, количество обработанных вершин, существование оптимального пути и его длину.
  \item Изучить и сравнить алгоритм A* и алгоритм Дейкстры на основе визуализации их работы.
  \item Изменять размер карты и её топологию.
  \item Выбирать начальную и конечную точку, если они не являются препятствиями.
\end{itemize}

\subsubsection{Recomendations}
\begin{itemize}
  \item Имеется начальная карта с размещёнными препятствиями, начальной точкой и конечной точкой пути.
  \item Начальная и конечная точки должны находятся внутри карты.
  \item Конечная точка должна быть пустой.
\end{itemize}