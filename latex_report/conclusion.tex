\section{Заключение}
В работе были рассмотрены два распространённых алгоритма поиска кратчайшего пути: A* и алгоритм Дейкстры.

В данной реализации использовался простой граф с одинаковой стоимостью рёбер. На практике же чаще всего идеальных условий не бывает. В сети дорог существуют разные типы дорог, поэтому необходимо установить разный вес рёбер: например, для асфальтированных дорог меньше, для просёлочных больше. Можно динамически редактировать вес, например, чтобы учитывать пробки: чем больше пробка, тем больше вес. Главное, соблюдать одинаковую размерность весов.

Чаще всего требуется быстрая работа алгоритма в условиях огромных карт. В реализованной программе для хранения данных используется список, что замедляет работу алгоритма, так как каждый раз приходится искать вершину с наименьшим коэффициентом и удалять её сдвигом. Для увеличения скорости можно использовать, например, бинарную кучу, тогда необходимо просто будет извлекать первый элемент.

На практике поиск кратчайшего пути за оптимальное время очень важен не только в устройствах навигации, но и в коммуникационной сфере и даже в искусственом интеллекте.